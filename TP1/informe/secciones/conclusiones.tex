\section{Conclusiones}

En este trabajo, pudimos estudiar el comportamiento de 2 redes con una densidad de nodos bastante grande, y pudimos ver cuales son los nodos mas distinguidos y los protocolos predominantes de las mismas.\\

Vimos que mediante un simple análisis de la cantidad de información de cada nodo, se pueden obtener fácilmente los nodos mas concurridos. También vimos como el porcentaje de protocolos ARP no es tan pequeño en algunos casos, como por ejemplo en la red Laboral, en donde el porcentaje de paquetes ARP superaban al porcentaje de paquetes IPv6.\\

También pudimos hacer un análisis de cuanto tiempo es necesario capturar trafico de una red, para poder obtener una medida estable de la entropía. Se observo que la fluctuación de nodos en una rede Wifi puede ser muy grande, y esto provoca que la entropía varíe mucho, a diferencia de la entropía de protocolos, la cual suele ser bastante constante debido a que son algo intrínseco de las redes hoy en día. Debido a este motivo, también pudimos ver como la entropia en si de los distintos nodos conectados a una red, es mucho mayor a la entropía de los protocolos que corren en la misma, ya que estos presentan mucha menor incertidumbre.
