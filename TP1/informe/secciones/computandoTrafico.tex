\section{Primera Consigna: Caputando tr\'afico}\label{sec:metodos_1}

\subsection{Desarrollo}

\subsection{Fuentes de información}

\subsubsection{Protocolos}

En primer lugar, la cátedra planteo una fuente de información que nos ayude a distinguir entre los distintos protocolos que se pueden hallar en una red. La fuente de información es la siguiente:

\begin{center}
	$S_{t_i;t_f}$ = $\{s_1 \dotsb s_n\}$ siendo $s_i$ = $p_i.type / p_i \in P$ entre los instantes de tiempo $[t_i; t_f ]$.
\end{center}

y P esta definida de la siguiente manera:

\begin{center}
	$P_{t_i;t_f}$ = $\{p_1 \dotsb p_n\}$ siendo $p_i$ el i-esimo paquete transmitido en la red entre los instantes de tiempo $[t_i, t_f]$
\end{center}

Entonces, lo que se hace es, realizar una captura de la red, entre los instantes $t_i$ y $t_f$, y de esto nos quedamos únicamente con los campos \textit{type} del frame Ethernet. Así obtenemos una fuente de informacion donde cada simbolo es un protocolo utilizao en la red.

\subsubsection{Nodos}

Ahora, debemos idear una fuente de información, que nos permita poder distinguir los hosts involucrados en la captura de trafico, y a su ves, distinguir entre ellos, cuales son los mas concurridos y con mas apariciones en la red.\\

Para esto, tomemos la fuente de información $P$ vista en el anterior punto, y tratemos de adaptarla para poder obtener una fuente de información que nos permita encontrar hosts en la red, solo basándonos en paquetes ARP.
Tomemos el siguiente subconjunto de $P$: 

\begin{center}
	$\overline{P}_{t_i;t_f}$ = $\{p_1 \dotsb p_n\}$ $\forall$ $p_i$ $\in$ P $/$ $p_i.type = ARP$ entre los instantes de tiempo $[t_i; t_f ]$.
\end{center}

Con esta fuente de información, lo que hacemos es quedarnos únicamente con los paquetes ARP de toda la captura de trafico. En base a esto proponemos la siguiente fuente de información.

\begin{center}
	$S_1$ = $R_{t_i;t_f}$ = $\{r_{ai} \mid r_{ai} = \overline{p}_i[ARP].ip\_origen \}$ $\bigcup$ $\{r_{bi} \mid r_{bi} = \overline{p}_i[ARP].ip\_destino \}$ entre los instantes de tiempo $[t_i; t_f ]$.
\end{center}

Esto quiere decir, nos quedamos con las IPs origen y destino de la capa ARP del paquete (del cual sabemos que tiene una, porque los símbolos son tomados de $\overline{P}$). De esta forma, obtenemos una fuente de información donde cada simbolo es un hosts de la red. \\

Teniendo esto, podemos calcular, tanto la entropía de ambas fuentes, así como la probabilidad y cantidad de información de cada símbolo, tanto para $S_1$ como para $S_{t_i;t_f}$. La cantidad de información de cada símbolo \textit{s}, se calcula de la siguiente forma:

\begin{center}
	$I(s)$ = $-log_2(P(s))$
\end{center}

en donde $P(s)$ es la probabilidad de ocurrencia de s, en todo el espacio muestral S, en este caso, todos los símbolos emitidos por $S_1$ y $S_{t_i;t_f}$. Luego, la entropía de la fuente es la que se obtiene mediante la siguiente formula:

\begin{center}
	$H(S)$ = $\sum_{s \in S} P(s)*I(s)$
\end{center}



