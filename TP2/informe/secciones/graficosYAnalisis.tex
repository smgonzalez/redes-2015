\section{Segunda Consigna: Gráficos y Análisis}

Luego de la breve explicacion de como se han implementado las herramientas a utilizar, y de haber propuesto 2 rutas a analizar, procedemos a mostrar los resultados obtenidos de las mismas. Para poder realizar los siguientes analisis, lo que hicimos fue: Correr la herramienta \textit{traceroute} a lo largo del dia, cada media hora con un script, durante 12 horas.

\subsection{Rutas encontradas}

Primero, vamos a presentar las rutas encontradas para las IPs propuestas. Se presentaran en formato tabla, y se indicara la ubicacion de cada IP segun Geo IP Tool.

\subsubsection{Universidad de Helsinki (Finlandia)}

El siguiente cuadro muestra como el enlace submarino se debería encontrar entre Argentina y Italia. Esta ruta, fue el resultado de 27 corridas de \textit{traceroute} a lo largo del dia, por lo que es lo mas certero que da nuestra herramienta. En la ruta se ve como al llegar al continente europeo, las IPs van saltando entre IPs locales de Suecia y Finlandia, hasta llegar al host destino.\\

Una cosa a destacar, es la aparicion de varios $\Delta$RTT negativos, incluso habiendo promediado los RTT de 10 paquetes distintos, y que ademas como se puede ver, todos fueron exitosos, y solo hubo 1 cambio de IP al inicio del \textit{traceroute}, en donde luego de 8 intentos cambio la IP. Esto muestra como por mas de que se promedien varios RTT, la cercanía de estos nodos, hacen que los RTT varíen mucho. Para eso esta bueno tambien tener el valor del desvió standard de los intentos, asi tenemos una aproximación de entre que valores puede estar verdaderamente el RTT, con alta probabilidad.

\begin{table}[H]
	\centering
	\caption{Ruta para Universidad de Helsinki}
	\label{table:finlandia}
	\begin{tabular}{|l|l|l|l|l|l|l|}
		\hline
		TTL & IP & Intentos & RTT Promedio & Desvio Standard & Delta RTT & Ubicación \\ \hline 
		1 & 10.0.0.1 & 8 & 59.12ms & 15.03ms & 59.12ms & Router Local (Argentina) \\ \hline 
		2 & 10.24.128.1 & 10 & 55.40ms & 4.30ms & -3.73ms & Router Local (Argentina) \\ \hline 
		3 & 181.47.254.85 & 10 & 58.20ms & 6.09ms & 2.80ms & Argentina (Bs As) \\ \hline 
		4 & 195.22.220.93 & 10 & 60.10ms & 6.01ms & 1.90ms & Argentina (Tigre) \\ \hline 
		5 & 195.22.220.92 & 10 & 63.11ms & 7.41ms & 3.01ms & Argentina (Tigre) \\ \hline 
		\textbf{6} & \textbf{149.3.183.11} & \textbf{10} & \textbf{271.50ms} & \textbf{4.70ms} & \textbf{208.39ms} & \textbf{Italia} \\ \hline 
		7 & No hubo respuesta & - & - & - & - & - \\ \hline 
		8 & 109.105.97.126 & 10 & 324.10ms & 2.60ms & 52.60ms & Suecia \\ \hline 
		9 & 109.105.102.102 & 10 & 336.80ms & 8.99ms & 12.70ms & Suecia \\ \hline 
		10 & 109.105.102.103 & 10 & 333.20ms & 17.01ms & -3.60ms& Suecia \\ \hline 
		11 & 193.167.253.9 & 10 & 333.80ms & 8.51ms & 0.60ms & Finlandia \\ \hline 
		12 & 128.214.173.242 & 10 & 336.40ms & 3.44ms & 2.60ms & Finlandia (Helsinki) \\ \hline 
		13 & 128.214.173.10 & 10 & 333.22ms & 12.11ms & -3.18ms & Finlandia (Helsinki) \\ \hline 
		14 & 128.214.189.85 & 10 & 328.40ms & 5.15ms & -4.82ms & Finlandia (Helsinki) \\ \hline 
		15 & 128.214.189.90 & 10 & 337.20ms & 5.81ms & 8.80ms & Finlandia (Helsinki) \\ \hline
	\end{tabular}
\end{table}

\subsubsection{Universidad de Oxford (Inglaterra)}

A diferencia de la la ruta vista en el punto anterior, esta resulto tener algunas particularidades. Para empezar, el enlace que tiene el mayor $\Delta$RTT, tiene ambos extremos en Estados Unidos, uno en Virginia y el otro en Kansas (segun Geo IP Tool y IP Location). Por otro lado, hay otro salto grande que se observa, que es entre el hop 7 y 9, y aquí si ambos extremos estan en distintos continentes. Nunca conseguimos que el hop 8 responda el paquete ping, ya que muy probablemente sea un router que descarta los paquetes de tipo ICMP. Esto es un comportamiento bastante habitual que tiene algunos routers.

De todas maneras, según se ve, el enlace debe estar entre Estados Unidos e Inglaterra. Probando tambien para otras IPs aledaneas a la utilizada, obtuvimos los mismo resultados, por lo que optamos por igual quedarnos con esta.

\begin{table}[H]
	\centering
	\caption{Ruta para Universidad de Oxford}
	\label{table:finlandia}
	\begin{tabular}{|l|l|l|l|l|l|l|}
		\hline
		TTL & IP & Intentos & RTT Promedio & Desvio Standard & Delta RTT & Ubicación \\ \hline 
		1 & 10.0.0.1 & 8 & 55.38ms & 10.89ms & 55.38ms & Router Local (Argentina) \\ \hline 
		2 & 10.24.128.1 & 10 & 60.00ms & 3.77ms & 4.62ms & Router Local (Argentina) \\ \hline 
		3 & 181.47.254.85 & 10 & 58.50ms & 5.78ms & -1.50ms & Argentina (Bs As) \\ \hline 
		4 & 208.178.195.214 & 10 & 65.30ms & 5.85ms & 6.80ms & Estados Unidos (Virginia) \\ \hline 
		5 & 208.178.195.213 & 10 & 65.10ms & 6.72ms & -0.20ms & Estados Unidos (Virginia) \\ \hline 
		6 & 67.17.75.66 & 10 & 182.60ms & 14.40ms & 117.50ms & Estados Unidos (Kansas) \\ \hline 
		7 & 4.68.111.121 & 10 & 188.30ms & 5.23ms & 5.70ms & Estados Unidos (Chicago) \\ \hline 
		8 & No hubo respuesta & - & - & - & - & - \\ \hline 
		\textbf{9} & \textbf{212.187.139.166} & \textbf{10} & \textbf{273.50ms} & \textbf{7.68ms} & \textbf{85.20ms} & \textbf{Inglaterra (Londres)} \\ \hline 
		10 & 146.97.33.2 & 10 & 273.10ms & 6.92ms & -0.40ms & Inglaterra (Londres) \\ \hline 
		11 & 146.97.37.194 & 10 & 276.70ms & 7.89ms & 3.60ms & Inglaterra (Londres) \\ \hline 
		12 & 193.63.108.94 & 10 & 273.40ms & 4.48ms & -3.30ms & Inglaterra (Gales) \\ \hline 
		13 & 193.63.108.98 & 10 & 273.90ms & 5.88ms & 0.50ms & Inglaterra (Gales) \\ \hline 
		14 & 193.63.109.42 & 10 & 271.20ms & 5.51ms & -2.70ms & Inglaterra (Gales) \\ \hline 
		15 & 192.76.21.71 & 10 & 273.70ms & 5.31ms & 2.50ms & Inglaterra (Oxford) \\ \hline 
		16 & 192.76.22.200 & 10 & 275.90ms & 5.15ms & 2.20ms & Inglaterra (Oxford) \\ \hline 
		17 & 192.76.32.62 & 10 & 283.50ms & 37.29ms & 7.60ms & Inglaterra (Oxford) \\ \hline 
		18 & 129.67.242.154 & 10 & 270.90ms & 4.20ms & -12.60ms & Inglaterra (Oxford) \\ \hline
	\end{tabular}
\end{table}

\subsection{Test de Grubbs}

En los resultados que mostramos anteriormente, se vio el $\Delta$RTT del enlace submarino para una sola corrida del \textit{traceroute}, pero esto fue solo para dar un primer vistaso y una primera estimacion sobre donde podria encontrarse el enlace submarino. Lo que ahora queremos mostrar, es el resultado que dio el Test de Grubbs, sobre los enlaces anteriormente encontrados, y ver si efectivamente son outliers, y potenciales enlaces submarinos.\\

ACLARACION: Si bien las tablas no es un buen metodo para presentar datos, en esta ocasión decidimos si utilizarla, ya que se pueden ver todos los valores para todas las corridas del test de Grubbs, esto son, el p-valor del \textit{normaltest}, el estidistico G y el valor para la condicion de rechazo C. \\

Como se puede ver, el test de grubbs indica que los $\Delta$RTT del enlace visto en el punto anterior (desde 195.22.220.92 a 149.3.183.11), son outliers de la muestra. Esto reafirma el hecho de que aqui se encuentre un enlace submarino. Tambien se ve como en 1 solo caso, el test indico que no hay outliers en la muestra, y ademas, el $\Delta$RTT mas grande, no es el enlace visto anteriormente. \\

Si se realiza una mirada mas de cerca sobre los resultados de esta ejecución, se podrá ver como tiene varios $\Delta$RTT muy grandes en comparación a otras ejecuciones. De echo esto hace que el \textit{normaltest} indique que es una distribución normal, pero sin embargo, el test de grubbs indica que el máximo valor no es un outlier. Debido a estas inconsistencias con respecto a las demás ejecuciones, se opto por ignorar esta ejecución, y asumir que fue influenciada por el estado de la red particular de ese momento.

\subsubsection{Universidad de Helsinki (Finlandia)}

\begin{table}[H]
	\centering
	\caption{Grubbs para Universidad de Finlandia}
	\label{table:grubbs-finlandia}
	\begin{tabular}{|l|l|l|l|l|l|l|l|}
		\hline
		HORA & IP Extremo 1 & IP Extremo 2 & DRTT & p-valor & G & C & Es outlier? \\ \hline 
19:57 & 128.214.173.10 & 128.214.189.85 & 333.70 & 0.588223 & 1.54 & 15.895 & NO \\ \hline
20:30 & 195.22.220.92 & 149.3.183.11 & 212.10 & 0.000000 & 3.25 & 15.895 & SI \\ \hline
20:30 & 195.22.220.92 & 149.3.183.11 & 209.70 & 0.000000 & 3.25 & 15.895 & SI \\ \hline
21:01 & 195.22.220.92 & 149.3.183.11 & 216.70 & 0.000000 & 3.26 & 15.895 & SI \\ \hline
21:33 & 195.22.220.92 & 149.3.183.11 & 214.30 & 0.000001 & 3.22 & 15.895 & SI \\ \hline
22:05 & 195.22.220.92 & 149.3.183.11 & 207.10 & 0.000001 & 3.20 & 15.895 & SI \\ \hline
22:36 & 195.22.220.92 & 149.3.183.11 & 207.60 & 0.000001 & 3.23 & 15.895 & SI \\ \hline
23:08 & 195.22.220.92 & 149.3.183.11 & 210.00 & 0.000001 & 3.21 & 15.895 & SI \\ \hline
23:39 & 195.22.220.92 & 149.3.183.11 & 207.70 & 0.000001 & 3.23 & 15.895 & SI \\ \hline
00:11 & 195.22.220.92 & 149.3.183.11 & 210.60 & 0.000000 & 3.25 & 15.895 & SI \\ \hline
00:42 & 195.22.220.92 & 149.3.183.11 & 213.30 & 0.000000 & 3.24 & 15.895 & SI \\ \hline
01:13 & 195.22.220.92 & 149.3.183.11 & 208.39 & 0.000000 & 3.25 & 15.895 & SI \\ \hline
01:45 & 195.22.220.92 & 149.3.183.11 & 210.00 & 0.000001 & 3.24 & 15.895 & SI \\ \hline
02:48 & 195.22.220.92 & 149.3.183.11 & 206.20 & 0.000001 & 3.22 & 15.895 & SI \\ \hline
03:19 & 195.22.220.92 & 149.3.183.11 & 210.70 & 0.000001 & 3.22 & 15.895 & SI \\ \hline
03:51 & 195.22.220.92 & 149.3.183.11 & 206.30 & 0.000000 & 3.25 & 15.895 & SI \\ \hline
04:22 & 195.22.220.92 & 149.3.183.11 & 206.30 & 0.000001 & 3.21 & 15.895 & SI \\ \hline
04:54 & 195.22.220.92 & 149.3.183.11 & 209.70 & 0.000000 & 3.25 & 15.895 & SI \\ \hline
05:25 & 195.22.220.92 & 149.3.183.11 & 209.70 & 0.000001 & 3.23 & 15.895 & SI \\ \hline
05:57 & 195.22.220.92 & 149.3.183.11 & 211.00 & 0.000000 & 3.24 & 15.895 & SI \\ \hline
06:28 & 195.22.220.92 & 149.3.183.11 & 205.90 & 0.000000 & 3.24 & 15.895 & SI \\ \hline
07:00 & 195.22.220.92 & 149.3.183.11 & 205.30 & 0.000000 & 3.23 & 15.895 & SI \\ \hline
07:31 & 195.22.220.92 & 149.3.183.11 & 214.00 & 0.000000 & 3.24 & 15.895 & SI \\ \hline
08:02 & 195.22.220.92 & 149.3.183.11 & 227.70 & 0.000000 & 3.25 & 15.895 & SI \\ \hline
08:34 & 195.22.220.92 & 149.3.183.11 & 205.70 & 0.000001 & 3.23 & 15.895 & SI \\ \hline
09:05 & 195.22.220.92 & 149.3.183.11 & 209.00 & 0.000001 & 3.23 & 15.895 & SI \\ \hline
09:37 & 195.22.220.92 & 149.3.183.11 & 217.70 & 0.000002 & 3.17 & 15.895 & SI \\ \hline
	\end{tabular}
\end{table}

\subsubsection{Universidad de Oxford (Inglaterra)}

En este caso, el test de Grubbs dio que el enlace con mayor $\Delta$RTT era un outlier, pero como vimos con las herramientas de geolicalizacion de IPs, este enlace probablemente no sea un enlace submarino. Debido a esto, optamos por volver a ejecutar el test de Grubbs, pero esta vez quitando este outlier, para tratar de identificar la presencia de un segundo outlier, y asi si ver si segun el test, el enlace que nosotros creemos que es el enlace submarino, es tambien un outlier segun el test de Grubbs. Los resultados pueden verse en la tabla \ref{table:grubbs-inglaterra}

\begin{table}[H]
	\centering
	\caption{Grubbs para Universidad de Finlandia}
	\label{table:grubbs-inglaterra}
	\begin{tabular}{|l|l|l|l|l|l|l|l|}
		\hline
		HORA & IP Extremo 1 & IP Extremo 2 & DRTT & p-valor & G & C & Es outlier? \\ \hline
		20:31 & 4.68.111.121 & 212.187.139.166 & 86.90 & 0.100648 & 2.48 & 15.971 & SI \\ \hline
		21:02 & 4.68.111.121 & 212.187.139.166 & 91.40 & 0.000021 & 3.02 & 15.971 & SI \\ \hline
		21:34 & 4.68.111.121 & 212.187.139.166 & 80.90 & 0.000311 & 2.85 & 15.971 & SI \\ \hline
		22:05 & 4.68.111.121 & 212.187.139.166 & 73.80 & 0.001433 & 2.69 & 15.971 & SI \\ \hline
		22:37 & 4.68.111.121 & 212.187.139.166 & 85.00 & 0.000052 & 2.95 & 15.971 & SI \\ \hline
		23:09 & 4.68.111.121 & 212.187.139.166 & 92.70 & 0.000007 & 3.12 & 15.971 & SI \\ \hline
		23:40 & 4.68.111.121 & 212.187.139.166 & 85.10 & 0.000090 & 2.92 & 15.971 & SI \\ \hline
		00:43 & 4.68.111.121 & 212.187.139.166 & 86.00 & 0.000011 & 3.01 & 15.971 & SI \\ \hline
		01:14 & 4.68.111.121 & 212.187.139.166 & 83.10 & 0.000012 & 3.02 & 15.971 & SI \\ \hline
		01:46 & 4.68.111.121 & 212.187.139.166 & 88.20 & 0.000008 & 3.08 & 15.971 & SI \\ \hline
		02:17 & 4.68.111.121 & 212.187.139.166 & 82.30 & 0.000008 & 3.03 & 15.971 & SI \\ \hline
		02:49 & 4.68.111.121 & 212.187.139.166 & 80.30 & 0.000110 & 2.89 & 15.971 & SI \\ \hline
		03:20 & 4.68.111.121 & 212.187.139.166 & 81.40 & 0.000015 & 2.99 & 15.971 & SI \\ \hline
		03:52 & 4.68.111.121 & 212.187.139.166 & 81.57 & 0.000021 & 3.00 & 15.971 & SI \\ \hline
		04:23 & 4.68.111.121 & 212.187.139.166 & 88.70 & 0.000120 & 2.92 & 15.971 & SI \\ \hline
		04:55 & 4.68.111.121 & 212.187.139.166 & 82.30 & 0.000048 & 2.96 & 15.971 & SI \\ \hline
		05:26 & 4.68.111.121 & 212.187.139.166 & 85.20 & 0.000011 & 3.05 & 15.971 & SI \\ \hline
		05:58 & 4.68.111.121 & 212.187.139.166 & 84.70 & 0.000014 & 3.00 & 15.971 & SI \\ \hline
		06:29 & 4.68.111.121 & 212.187.139.166 & 84.30 & 0.000019 & 2.99 & 15.971 & SI \\ \hline
		08:03 & 4.68.111.121 & 212.187.139.166 & 108.10 & 0.000178 & 2.97 & 15.971 & SI \\ \hline
		08:34 & 193.63.108.98 & 193.63.109.42 & 141.50 & 0.042960 & 2.28 & 15.971 & SI \\ \hline
		09:06 & 4.68.111.121 & 212.187.139.166 & 87.30 & 0.000010 & 3.04 & 15.971 & SI \\ \hline
		09:38 & 4.68.111.121 & 212.187.139.166 & 85.30 & 0.000018 & 2.98 & 15.971 & SI \\ \hline
	\end{tabular}
\end{table}